% \addcontentsline{toc}{chapter}{Abstract}
\chapter*{Abstract}
\gls{so} is today for many developers a well known \gls{qa} system. 
However, \gls{so} has a high requirement to the questions and answers posted, which is reflected through their voting and reputation system. 
This peer-review processes can be used as an indicator to a questions quality, where questions with high up-votes can be defined as good questions.
In this thesis, a system has been developed using  \gls{ml} and \gls{svm} to see if it is possible to predict whether or not a new question will be 
considered a good or bad question. 
\vspace{0.5em}\newline
This was achieved by using the \gls{se} data set, specifically using the one for \gls{so}. 
Questions were dived into two classes, where bad questions was question with a vote score below zero, and good questions were those above zero. 
Based on content in the various questions, a set of feature detectors was developed and tested against the raw data set. 
Surprisingly, the features actually lowered the accuracy score.
The raw, unprocessed classifier achieved a score of 79.90\%.
The classifier using Porter stemming and all features achieved a score of 75.97\%, and the classifier without stemming using all the features got a score of 79.12\%.

\hypersetup{pageanchor=false}