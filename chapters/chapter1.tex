%\chapter{Introduction}
\label{chap:introduction}

Today, many uses the Internet as a resource to find answers to their questions and problems. 
In the past, one were often restricted to only use keywords and not being able to pose the problem as you would when asking another human being. 
Most search engines today can handle natural language queries, which makes it easier to find the answer you are looking for. 
The Internet offers a wide range of resources to acquire new knowledge, everything from encyclopaedias to blogs, forums and \gls{qa} communities.
One well known \gls{qa} community is the \gls{se} community, which is built upon the same model as \gls{so} \cite{Ahmed2015}.
\gls{se} has grown large since its release in 2009, and now contains 154 different communities.
\vspace{0.5em}\newline
As a developer, one often find oneself in the situation that a part of the code does not work, you get weird error messages, or you are simply stuck. 
This is were \gls{so} comes in. \gls{so} is a part of the \gls{se} community, although \gls{so} was actually released before \gls{se}. 
Jeff Atwood and Joel Spolsky wanted to offer programmers a \gls{qa} site where they could get the answer they wanted without having to read through a lot of text, 
see others posting "I also have the same issue" or having to subscribe and pay to see the solution \cite{Spolsky2008}. 
Question (and answer) quality is maintained through the use of a peer-reviewed gamification system, where users are awarded with votes, reputation and badges for their 
participation \cite{Movshovitz-Attias2013, M.Sewak2010, Spolsky2008, Treude2011}. 
One of the requirements is that the questions should be of good quality 
\cite{StackOverflow.com2016a, StackOverflow.com2016f, StackOverflow.com2016e}.
If a question is bad, users can vote to close or delete it (in which the question will be put on hold). 
A question can be put on hold or closed if they meet any of the following criterias: 
Exact duplicate (same question has been asked before), off-topic (not related to \gls{so}), unclear what is being asked, too broad (e.g. could write a book about question being asked) 
or primarily opinion-based \cite{CommunityWiki2016b, StackOverflow.com2016b}.



\section{Problem description}
\label{sec:problem_description}
Most of the systems that have been developed so far focuses on finding the best answer to a question asked by the user. 
Few, if any, focuses on the quality of the question being asked. 
What defines a good question, and can we in anyway predict whether or not a new question posted on \gls{so} will be considered good or bad by the community?
There are many users who has either a negative view or relationship in regards to \gls{so}.
Many experience that their questions gets down-voted, closed or even deleted. 
For some, they simply do not know how to ask an acceptable question.
Questions related to homework are one example of questions that are not accepted on \gls{so}.
There is even a post on Meta.StackExchange discussing whether or not it should be acceptable to use greetings and sentiments in posts \cite{CommunityWiki2016a}.
Therefore, the question becomes: What is and is not a valid question on \gls{so}?
%What if there was a system that could give a prediction on the quality of your question?

\section{Research questions}
\label{sec:research_questions}

\begin{itemize}
	\item What defines a good (coding) question on \gls{so}?
	\item Can we predict a questions quality by using \gls{svm}?
	\item What type of features increases the accuracy of the \gls{svm}?
\end{itemize}

\section{Methodology to be used}
\label{sec:methodology_to_use}
The theoretical background in this thesis is mainly focused on \gls{qc} and similar research in relation to \gls{so}.
What has been the focus of other researchers, and in what way did they proceed to solve their questions?
The analysis of the questions are done by using the publicly available database dump, which is available via \gls{se} 
archive\footnote{StackExchange dataset: \url{https://archive.org/details/stackexchange} \\ (Downloaded 30. March 2016).} \cite{StackExchange2016}. 
There are several others who have used the same dataset \cite{Ahmed2015, Anderson2012, Hanrahan2012, Movshovitz-Attias2013, Posnett2012, Short2014, Stanley2013, Yang2014}.
Taking into consideration that \gls{so} was released in 2008, it means that it now contains approximately 8 years of peer-reviewed data.
Because of the size of the data set, and the total amount of posted questions, going through all questions manually would be too time-consuming.
Therefore only a select few were studied too see if it was possible to identify what separated the highly up and down-voted questions.
\vspace{0.5em}\newline
The goal was to develop a \gls{ml} learning system which was based on \gls{svm}, since many papers document that this has the best classification accuracy for text classification. 
The methodology therefore also includes a documentation on the development process, and how and why the given features used were selected.
\vspace{0.5em}\newline
For the sake of replicability, and also be able to undo potential errors, the system is available in a a GitHub repository\footnote{
	GitHub repository: 
	\url{https://github.com/klAndersen/IMT4904\_MasterThesis\_Code}
}. 
In addition to the source code, the repository also contains both the samples that was used (stored in CSV files), and the models that was created. 

% \todo[size=\small]{add cites here}~

\section{Justification, Motivation and Benefits}
\label{sec:justification}
Many systems focuses only on finding a good answer, and does not ask if it is a good question.
As a famous Norwegian saying goes\footnote{
	Although its origin comes from a Danish word collection from 1682: \\
	\url{https://snl.no/En\_d\%C3\%A5re\_kan\_sp\%C3\%B8rre\_mer\_enn\_ti\_vise\_kan\_svare}.
}: "A fool may ask more than ten wise men can answer".
This means that new research possibilities could be opened up in relation to researching question quality by expanding the system. 
Since all the communities within \gls{se} is based on the same model, few modifications would be needed to scale the program to be used within the other communities.
As noted in several papers \cite{Movshovitz-Attias2013, Nasehi2012, Posnett2012, M.Sewak2010, Treude2011, Yang2014}, question quality is measured based on the amount of votes given. 
Which can also be compared against the peer-review process in academia, and given that \gls{so} targets professionals and experts, 
using \gls{so} as a scientific reference is not that unusual\footnote{
	 \textcite[p.~1]{Posnett2012} noted that \gls{so} "ranked 2$^{nd}$ among reference sites, 4$^{th}$ among computer science sites, and 97$^{th}$ overall among all websites".
	 }.
\gls{se} has also been the focus of various researchers these past years \cite{Vasilescu2012}.
Improving ones own ability to ask better questions can also have a pedagogical effect, which means that this system could be implemented in education. 

\begin{comment}
The quality of a question is based on the votes given by the users of the StackOverflow community. 
Therefore the vote score was used to label the question as either good (score > 0) or bad (score < 0). 
This is justified by the fact that you can draw a comparison between the peer-review process used in academia and the peer-review process used in StackOverflow. 
The process starts with someone posting a question, which is then read, scored and given feedback (through comments or answers). 
The feedback can then improve the question through edits by the poster, based on the feedback given 
(e.g. adding additional information or explaining how the given problem(s) occurred).
\vspace{0.5em}\newline
Scikit-learns \gls{svm} implementation is based on libsvm \cite{Chang2011}, but it is simpler to use. 
In addition, scikit-learn focuses on code quality\footnote{Coverage on their GitHub repository was 94\% on 06. May 2015.}, 
to ensure that everything works as it should \cite[p.~3]{Pedregosa2011}.
\vspace{0.5em}\newline
It is not a lot of research being done related to asking and defining good questions. 
Although the focus is only on programming questions, it can still be useful to improve question quality. 
Even if you cannot predict if it is a good question, you can still avoid asking a bad questions. 
This in turn can help to improve the question quality (at least acceptability), and avoid asking questions that will be closed or ignored on StackOverflow.
\end{comment}

\section{Limitations}
\label{sec:limitations}
The selection of questions is only 20,000 (10,000 good and bad), which is a lower compared to some of the work done by others\footnote{
	\textcite{Wang2013} used 63,863 unique questions, %  to analyse how developers posted and answered questions.
	\textcite{Anderson2012} used 28,722 questions % when analysing the long term value for questions and if a question would get an accepted answer.
	and \textcite{Treude2011} used 38,419 questions. % for question categorization.
}.
The retrieval of questions could also have been better, since the vote score was based on static values, rather than selecting those with the highest/lowest score.
Some features were not very representative (e.g. Hexadecimal, which only occurred in 160 of the 20,000 questions), and would therefore have been excluded\footnote{
	The minimum document frequency was set to 0.01, meaning it ignores words that appear in less than 1\% of the questions.
}.
The version of Scikit-learn that was used was the latest development version (v0.18.dev0), instead of the stable. 
This means that potential bugs and un-finished implementations can have an effect on the prediction and probability (e.g. giving the wrong results).
A limitation is also that the focus is only on \gls{so}, meaning that some features that provide a good accuracy score for \gls{so} may have no impact in other communities.
 
\section{Thesis contribution}
\label{sec:thesis_contribution}
This thesis contribution can be summarized as to the following: 
Predicting programming question quality by  using \gls{ai} and \gls{ml} to improve the questions quality. 
Instead of posting bad questions that can get down-voted or closed, the developed system could be able to give feedback to the user whether or not it is a good question.
Furthermore, the research presented could open up for new research in relation to how we ask questions online, and in what ways these best can be analysed.
It can also be used for educational purposes, e.g. having users iteratively improve their question quality by asking the system questions.

\section{Thesis structure}
\label{sec:thesis_structure}
The thesis is structured as follows. In Chapter \ref{chap:chapter2}, relevant research is presented. 
This includes \gls{so}, a definition of what a question is and short about \gls{svm}, to give an overview of the current state. 
The thesis continues with a presentation of the methodology that was used in Chapter \ref{chap:chapter3}. 
This includes information on the data set, the created database and how the development progressed. 
A short explanation to the selection of feature detectors is also included. 
In Chapter \ref{chap:chapter4}, the results are presented.
Following is a discussion on the choices that were made, the issues that occurred and the results, which is presented in Chapter \ref{chap:chapter5}.
The thesis ends with a conclusion and suggestions for further work in Chapter~\ref{chap:chapter6}.
