%\chapter{Discussions}
\label{chap:chapter5}




For Discussion chapter:
--------------------

% Re-write and use what ever fits
In the \gls{bow} model, only single words or word stems are used as features for representing document content. 
The issue is that learning algorithms are restricted to detecting patterns in the used terminology only, while ignoring conceptual patterns. 
List of weaknesses with using \gls{bow} (1-3 addressed issues on a lexical level, and 4 conceptual level):
\begin{enumerate}
	\item Multi-Word Expressions with an own meaning like "European Union" are chunked into pieces with possibly very different meanings like "union".
	\item Synonymous Words like "tungsten" and "wolfram" are mapped into different features.
	\item Polysemous Words are treated as one single feature while they may actually have multiple distinct meanings.
	\item Lack of Generalization: there is no way to generalize similar terms like "beef" and "pork" to their common hypernym "meat".
\end{enumerate}
WordNet database organizes simple words and multi-word expressions of different syntactic categories into so called synonym sets (synsets), 
each of which represents an underlying concept and links these through semantic relations. \\
Conceptual Document Representation:
\begin{itemize}
	\item Candidate Term Detection: Strategy built on the assumption that if you find the longest multi-word expressions in the text, 
	the lexicon will lead to a mapping to the most specific concept for that word (instead of querying single words, which may lead to wrong mapping).
	\item Syntactical Patterns:Analysis by using POS-tagging.
	\item Morphological Transformations: Entry form, base form reduction. 
	Stemming if the first query for the inflected forms on the original lexicon turned out unsuccessful.
	\item Word Sense Disambiguation (WSD): A lexical entry for an expression does not necessarily imply a one-to-one mapping to a concept in the ontology. 
	\item Disambiguate an expression versus multiple possible concepts.
	\item Generalization: Going from specific concepts in the text to general concept representations. Mapping words based on generalization (up to a certain level).
\end{itemize}
\cite{Bloehdorn2004}

% Re-write and use what ever fits
\gls{qc}: predict the entity type of the answer of a natural language question, mostly achieved by using machine learning. 
Used Latent Semantic Analysis (LSA) technique to reduce the large feature space of questions to a much smaller and efficient feature space. 
Two different classifiers: Back-Propagation Neural Networks (BPNN) and Support Vector Machines (SVM). 
Found that using LSA on question classification made it more time efficient and improved classification accuracy by removing redundant features. 
Discovered that when the original feature space is compact and efficient, its reduced space performs better than a large feature space with a rich set of features. 
They also found that in the reduced feature space, BPNN was better than SVM.  
Competitive with state-of-the-art, even though they used smaller feature space.
\cite{Loni2011}

------------------------

Note to self: Map graph over feature impact (unprocessed, singular, all)
Also add in estimated training time for exhaustive search (e.g. ~120 minutes for SVC vs ~100 for SGD over 16k questions (since 4k = test)). 


% could be re-phrased into "in this paper, they define question analysis with two different approaches..."
In the paper by \textcite{Toba2011}, they experiment with the use of statistical learning to find the expected answer pattern for factoid \gls{qa} pairs. 
E.g. if you ask someone where a certain event took place, the answer pattern would be a location. 
They group question analysis into two approaches; pattern-based (high precision, low recall) and \gls{ml} (high recall, 
low precision\footnote{
	Low precision can occur if the feature sets are not fitted well enough during classifier training 
	\cite[p.~283]{Toba2011}.
}). 
Pattern-based would match word sequences against a set of patterns (e.g. regular expressions), whereas \gls{ml} would be based on the accuracy of the classifier 
(e.g. lexical or linguistic feature sets). 
% editor space
The retrieval of \gls{qa} pairs is done by using a statistical relation framework: Bayesian Analogical Reasoning (BAR). 
Features sets are extracted from the training set by use of binary values checking if the question contains a given question word. 
The BAR framework then learns the related features and computes the estimation for them. 
Thereafter \gls{qa} pairs are retrieved from the testing set and compared against the training set. 
Afterwards, the \gls{qa} pairs that have identical question words are identified, and overlapping pairs are grouped according their named entity group.
To retrieve named entities, they used two different recognizers. 
The first was Stanford (extracts the person, organization and location), and the second was dictionary based (extract number entities and fine-grained noun-based entities). 
Question words were extracted by building a question word list from the training set (achieved by using Stanford Part-of-Speech (POS) tagger). 
Then for each question, look for the appearance of the question word to create the feature set.
Mapped named-entities.
\cite{Toba2011}

Potentially move "all this failed and went wrong" here

~\\
To write: \\
Tutorials that I went through \\
Using SGD (based on tutorials from scikit-learn) \\
Testing out different text classification algorithms (SVC, SGD and LinearSVC) \\
Paths, Windows vs. Linux, parallellization, gpu\_count, etc

\section{Data and Testing}
\label{sec:data_and_testing}
discussion on the data set and how it was tested. \\
the results and what they showed.  \\
potential improvements, etc.

\section[Artificial Intelligence Methods]{\glsentrylong{ai} (\glsentryshort{ai}) Methods}
\label{sec:ai_methods}
alternative methods and options (e.g. one could have used ann or k-nn, but as shown in\ldots) \\
not sure if this section is relevant?

\section{Implementation Architecture}
\label{sec:implemented_architecture}
discussion on the code that was written and its functionality \\
what worked, what should be updated/changed, etc.

\section{Limitations and other issues}
\label{sec:limitations_and_issues}
why didn't something work as intended? \\
why wasn't X completed/implemented? \\
etc.
